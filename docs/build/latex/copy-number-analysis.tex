%% Generated by Sphinx.
\def\sphinxdocclass{article}
\documentclass[letterpaper,10pt,english]{sphinxhowto}
\ifdefined\pdfpxdimen
   \let\sphinxpxdimen\pdfpxdimen\else\newdimen\sphinxpxdimen
\fi \sphinxpxdimen=.75bp\relax

\PassOptionsToPackage{warn}{textcomp}
\usepackage[utf8]{inputenc}
\ifdefined\DeclareUnicodeCharacter
% support both utf8 and utf8x syntaxes
  \ifdefined\DeclareUnicodeCharacterAsOptional
    \def\sphinxDUC#1{\DeclareUnicodeCharacter{"#1}}
  \else
    \let\sphinxDUC\DeclareUnicodeCharacter
  \fi
  \sphinxDUC{00A0}{\nobreakspace}
  \sphinxDUC{2500}{\sphinxunichar{2500}}
  \sphinxDUC{2502}{\sphinxunichar{2502}}
  \sphinxDUC{2514}{\sphinxunichar{2514}}
  \sphinxDUC{251C}{\sphinxunichar{251C}}
  \sphinxDUC{2572}{\textbackslash}
\fi
\usepackage{cmap}
\usepackage[T1]{fontenc}
\usepackage{amsmath,amssymb,amstext}
\usepackage{babel}



\usepackage{times}
\expandafter\ifx\csname T@LGR\endcsname\relax
\else
% LGR was declared as font encoding
  \substitutefont{LGR}{\rmdefault}{cmr}
  \substitutefont{LGR}{\sfdefault}{cmss}
  \substitutefont{LGR}{\ttdefault}{cmtt}
\fi
\expandafter\ifx\csname T@X2\endcsname\relax
  \expandafter\ifx\csname T@T2A\endcsname\relax
  \else
  % T2A was declared as font encoding
    \substitutefont{T2A}{\rmdefault}{cmr}
    \substitutefont{T2A}{\sfdefault}{cmss}
    \substitutefont{T2A}{\ttdefault}{cmtt}
  \fi
\else
% X2 was declared as font encoding
  \substitutefont{X2}{\rmdefault}{cmr}
  \substitutefont{X2}{\sfdefault}{cmss}
  \substitutefont{X2}{\ttdefault}{cmtt}
\fi


\usepackage[Bjarne]{fncychap}
\usepackage{sphinx}

\fvset{fontsize=\small}
\usepackage{geometry}


% Include hyperref last.
\usepackage{hyperref}
% Fix anchor placement for figures with captions.
\usepackage{hypcap}% it must be loaded after hyperref.
% Set up styles of URL: it should be placed after hyperref.
\urlstyle{same}


\usepackage{sphinxmessages}
\setcounter{tocdepth}{4}
\setcounter{secnumdepth}{4}


\title{copy\sphinxhyphen{}number\sphinxhyphen{}analysis}
\date{Mar 25, 2021}
\release{1.0}
\author{Anibal Morales, IAEA PBGL \sphinxhyphen{} UN FAO}
\newcommand{\sphinxlogo}{\vbox{}}
\renewcommand{\releasename}{Version 1.0}
\makeindex
\begin{document}

\pagestyle{empty}

        \pagenumbering{Roman} %%% to avoid page 1 conflict with actual page
        \begin{titlepage}
            \vspace*{10mm} %%% * is used to give space from top
            \flushright\textbf{\Huge {PBGL CNV-seq Analysis v1.0\\}}
            \vspace{0mm} %%% * is used to give space from top
            \textbf{\Large {A Laboratory Manual\\}}
            \vspace{50mm}
            \textbf{\Large {Anibal E. Morales-Zambrana\\}}
            \vspace{10mm}
            \textbf{\Large {Plant Breeding and Genetics Laboratory\\}}
            \vspace{0mm}
            \textbf{\Large {FAO/IAEA Joint Division\\}}
            \vspace{0mm}
            \textbf{\Large {Seibersdorf, Austria\\}}
	    \vspace{10mm}
            \normalsize Created: March, 2021\\
            \vspace*{0mm}
            \normalsize  Last updated: 24 March 2021
            %% \vfill adds at the bottom
            \vfill
            \small\flushleft {{\textbf {Please note:}} \textit {This is not an official IAEA publication but is made available as working material. The material has not undergone an official review by the IAEA. The views
expressed do not necessarily reflect those of the International Atomic Energy Agency or its Member States and remain the responsibility of the contributors. The use of particular designations of countries or territories does not imply any judgement by the publisher, the IAEA, as to the legal status of such countries or territories, of their authorities and institutions or of the delimitation of their boundaries. The mention of names of specific companies or products (whether or not indicated as registered) does not imply any intention to infringe proprietary rights, nor should it be construed as an endorsement or recommendation on the part of the IAEA.}}
        \end{titlepage}
        \pagenumbering{arabic}
        \newcommand{\sectionbreak}{\clearpage}

\pagestyle{plain}
\sphinxtableofcontents
\pagestyle{normal}
\phantomsection\label{\detokenize{index::doc}}



\section{Background}
\label{\detokenize{index:background}}
\sphinxAtStartPar
\sphinxstylestrong{{[}DRAFT{]}}

\sphinxAtStartPar
Copy number variation (CNV) analysis using CNV\sphinxhyphen{}seq, R, Jupyter Notebooks, Miniconda3, and Git.

\begin{sphinxadmonition}{note}{Note:}
\sphinxAtStartPar
This is not an official IAEA publication but is made available as working material. The material has not undergone an official review by the IAEA. The views expressed do not necessarily reflect those of the International Atomic Energy Agency or its Member States and remain the responsibility of the contributors. The use of particular designations of countries or territories does not imply any judgement by the publisher, the IAEA, as to the legal status of such countries or territories, of their authorities and institutions or of the delimitation of their boundaries. The mention of names of specific companies or products (whether or not indicated as registered) does not imply any intention to infringe proprietary rights, nor should it be construed as an endorsement or recommendation on the part of the IAEA.
\end{sphinxadmonition}


\section{Installations}
\label{\detokenize{index:installations}}
\sphinxAtStartPar
Before installing any necessary software, it is recommended to check if the computer is running 32\sphinxhyphen{}bit or 64\sphinxhyphen{}bit for downloading Miniconda3. Run the following to verify the system:

\begin{sphinxVerbatim}[commandchars=\\\{\}]
\PYGZdl{} uname \PYGZhy{}m
\end{sphinxVerbatim}


\subsection{Miniconda3 (conda)}
\label{\detokenize{index:miniconda3-conda}}
\sphinxAtStartPar
Download the Miniconda3, or simply “conda”, installer:
\begin{itemize}
\item {} 
\sphinxAtStartPar
\sphinxhref{https://docs.conda.io/en/latest/miniconda.html\#linux-installers}{Miniconda3 installer for Linux}

\end{itemize}

\sphinxAtStartPar
Run the downloaded installer (for a 64\sphinxhyphen{}bit system):

\begin{sphinxVerbatim}[commandchars=\\\{\}]
\PYGZdl{} bash Miniconda3\PYGZhy{}latest\PYGZhy{}Linux\PYGZhy{}x86\PYGZus{}64.sh
\end{sphinxVerbatim}

\sphinxAtStartPar
Open a new terminal window for conda to take effect. Verify the installation in new terminal window with:

\begin{sphinxVerbatim}[commandchars=\\\{\}]
\PYGZdl{} conda list
\end{sphinxVerbatim}


\subsection{Git with conda}
\label{\detokenize{index:git-with-conda}}
\sphinxAtStartPar
Git will be installed first to clone locally (download a copy to your local computer) the CNV\sphinxhyphen{}seq repository from GitHub. To do so, run the following:

\begin{sphinxVerbatim}[commandchars=\\\{\}]
\PYGZdl{} conda install \PYGZhy{}c anaconda git
\end{sphinxVerbatim}

\sphinxAtStartPar
After the installation, clone the CNV\sphinxhyphen{}seq repository to the local computer in the desired directory.

\begin{sphinxVerbatim}[commandchars=\\\{\}]
\PYGZdl{} git clone https://github.com/amora197/copy\PYGZhy{}number\PYGZhy{}analysis.git
\end{sphinxVerbatim}

\sphinxAtStartPar
Verify that the installation is complete by listing the files in the directory.

\begin{sphinxVerbatim}[commandchars=\\\{\}]
\PYGZdl{} ls \PYGZhy{}l
\end{sphinxVerbatim}

\sphinxAtStartPar
A folder called \sphinxstylestrong{copy\sphinxhyphen{}number\sphinxhyphen{}analysis} should be listed in the directory.


\subsection{Required Libraries with conda}
\label{\detokenize{index:required-libraries-with-conda}}
\sphinxAtStartPar
CNV\sphinxhyphen{}seq has multiple dependencies, listed below:
\begin{itemize}
\item {} 
\sphinxAtStartPar
Git

\item {} 
\sphinxAtStartPar
R
\begin{itemize}
\item {} 
\sphinxAtStartPar
configr

\item {} 
\sphinxAtStartPar
ggplot2

\item {} 
\sphinxAtStartPar
BiocManager

\item {} 
\sphinxAtStartPar
Bioconductor\sphinxhyphen{}GenomicAlignments

\end{itemize}

\item {} 
\sphinxAtStartPar
Jupyter Notebook
\begin{itemize}
\item {} 
\sphinxAtStartPar
IRkernel

\end{itemize}

\end{itemize}

\sphinxAtStartPar
There are two ways to install the rest of the necessary libraries to run CNV\sphinxhyphen{}seq: automatically or manually. The former is slower, providing a long coffee break while the conda installations run. The latter proves a faster way to get the tool up\sphinxhyphen{}and\sphinxhyphen{}running.


\subsubsection{Automatically (slower)}
\label{\detokenize{index:automatically-slower}}
\sphinxAtStartPar
Two YAML files, \sphinxstylestrong{environment.yml} and \sphinxstylestrong{libraries.yml}, are provided to automatically create a conda environment and install the dependent libraries. The former creates the conda environment, along R, Jupyter Notebook, and the R\sphinxhyphen{}kernel in Jupyter. The latter installs dependent R libraries. Run \sphinxstylestrong{environment.yml}:

\begin{sphinxVerbatim}[commandchars=\\\{\}]
\PYGZdl{} conda env create \PYGZhy{}\PYGZhy{}file envs/environment.yml
\end{sphinxVerbatim}

\sphinxAtStartPar
Once done, the created environment can be verified running:

\begin{sphinxVerbatim}[commandchars=\\\{\}]
\PYGZdl{} conda env list
\end{sphinxVerbatim}

\sphinxAtStartPar
Activate the created environment (\sphinxstylestrong{cnv\sphinxhyphen{}seq}) and run \sphinxstylestrong{libraries.yml}:

\begin{sphinxVerbatim}[commandchars=\\\{\}]
\PYGZdl{} conda activate cnv\PYGZhy{}seq
\PYGZdl{} conda env update \PYGZhy{}\PYGZhy{}file envs/libraries.yml
\end{sphinxVerbatim}

\sphinxAtStartPar
Once done, all the necessary packages should be installed. This can be verified with:

\begin{sphinxVerbatim}[commandchars=\\\{\}]
\PYGZdl{} conda list
\end{sphinxVerbatim}


\subsubsection{Manually (faster)}
\label{\detokenize{index:manually-faster}}
\sphinxAtStartPar
To manually create and activate an environment, run:

\begin{sphinxVerbatim}[commandchars=\\\{\}]
\PYGZdl{} conda create \PYGZhy{}\PYGZhy{}name cnv\PYGZhy{}seq
\end{sphinxVerbatim}

\sphinxAtStartPar
Once done, the created environment can be verified running:

\begin{sphinxVerbatim}[commandchars=\\\{\}]
\PYGZdl{} conda env list
\end{sphinxVerbatim}

\sphinxAtStartPar
Activate the virtual environment with:

\begin{sphinxVerbatim}[commandchars=\\\{\}]
\PYGZdl{} conda activate cnv\PYGZhy{}seq
\end{sphinxVerbatim}

\sphinxAtStartPar
Start running the installations of the necessary libraries, paying attention to the prompts for each one:

\begin{sphinxVerbatim}[commandchars=\\\{\}]
\PYGZdl{} conda install \PYGZhy{}c conda\PYGZhy{}forge r\PYGZhy{}base=4.0
\PYGZdl{} conda install \PYGZhy{}c anaconda jupyter
\PYGZdl{} conda install \PYGZhy{}c r r\PYGZhy{}irkernel
\PYGZdl{} conda install \PYGZhy{}c conda\PYGZhy{}forge r\PYGZhy{}biocmanager
\PYGZdl{} conda install \PYGZhy{}c bioconda bioconductor\PYGZhy{}genomicalignments
\PYGZdl{} conda install \PYGZhy{}c pcgr r\PYGZhy{}configr
\PYGZdl{} conda install \PYGZhy{}c r r\PYGZhy{}ggplot2
\end{sphinxVerbatim}

\sphinxAtStartPar
Once done, all the necessary packages should be installed. This can be verified with:

\begin{sphinxVerbatim}[commandchars=\\\{\}]
\PYGZdl{} conda list
\end{sphinxVerbatim}


\section{Running a Jupyter Notebook}
\label{\detokenize{index:running-a-jupyter-notebook}}
\sphinxAtStartPar
To access the Jupyter Notebooks, run:

\begin{sphinxVerbatim}[commandchars=\\\{\}]
\PYGZdl{} jupyter notebook
\end{sphinxVerbatim}

\sphinxAtStartPar
This command will start a Jupyter Notebook session inside the directory the command is run. The user can navigate between directories, visualize files, and edit files in the browser by clicking on directories or files, respectively.

\sphinxAtStartPar
Look for the directory \sphinxstylestrong{copy\sphinxhyphen{}number\sphinxhyphen{}analysis} and click on it. Click on \sphinxstylestrong{jupyter\sphinxhyphen{}notebooks} directory, which contains four directories and two Jupyter Notebooks. Here is a breakdown of each:
\begin{itemize}
\item {} 
\sphinxAtStartPar
\sphinxtitleref{config}:
\begin{itemize}
\item {} 
\sphinxAtStartPar
directory containing configuration files

\end{itemize}

\item {} 
\sphinxAtStartPar
\sphinxtitleref{helper\sphinxhyphen{}functions}:
\begin{itemize}
\item {} 
\sphinxAtStartPar
directory containing R scripts with functions to calculate and plot CNVs

\end{itemize}

\item {} 
\sphinxAtStartPar
\sphinxtitleref{images}:
\begin{itemize}
\item {} 
\sphinxAtStartPar
directory that will contain output CNV plots after running CNV\sphinxhyphen{}seq

\end{itemize}

\item {} 
\sphinxAtStartPar
\sphinxtitleref{tab\sphinxhyphen{}files}:
\begin{itemize}
\item {} 
\sphinxAtStartPar
directory that will contain two types of output tab\sphinxhyphen{}delimited files:
\begin{itemize}
\item {} 
\sphinxAtStartPar
hits used to calculate CNVs

\item {} 
\sphinxAtStartPar
CNVs per chromosome per comparison

\end{itemize}

\end{itemize}

\item {} 
\sphinxAtStartPar
two Jupyter Notebooks:
\begin{itemize}
\item {} 
\sphinxAtStartPar
RCNV\sphinxhyphen{}seq\sphinxhyphen{}example.ipynb
\begin{itemize}
\item {} 
\sphinxAtStartPar
example analysis of sorghum

\end{itemize}

\item {} 
\sphinxAtStartPar
RCNV\sphinxhyphen{}seq\sphinxhyphen{}template.ipynb
\begin{itemize}
\item {} 
\sphinxAtStartPar
template for the user

\end{itemize}

\end{itemize}

\end{itemize}

\begin{sphinxadmonition}{note}{Note:}
\sphinxAtStartPar
Jupyter lets the user duplicate, rename, move, download, view, or edit files in a web browser. This can be done by clicking the box next to a file and choosing accordingly.
\end{sphinxadmonition}


\section{Editing the Configuration File}
\label{\detokenize{index:editing-the-configuration-file}}
\sphinxAtStartPar
In order to run the CNV\sphinxhyphen{}seq Jupyter Notebook, the user needs to feed it with a configuration file (\sphinxstylestrong{config\sphinxhyphen{}CNVseq.yml}) that specifies the paths to the bam files, comparisons to be done, chromosomes to analyze, and parameter definitions for calculating and plotting CNVs.

\sphinxAtStartPar
The configuration file \sphinxstylestrong{config\sphinxhyphen{}CNVseq.yml} can be found in the \sphinxstylestrong{copy\sphinxhyphen{}number\sphinxhyphen{}analysis/jupyter\sphinxhyphen{}notebooks/config} directory.

\begin{sphinxadmonition}{note}{Note:}
\sphinxAtStartPar
The user needs to edit \sphinxstylestrong{config\sphinxhyphen{}CNVseq.yml} to point towards bam/bed files; specify comparisons and chromosomes to analyze; and define the parameters to calculate/plot CNVs.
\end{sphinxadmonition}

\sphinxAtStartPar
Two example configuration files are provided (\sphinxstylestrong{example1\sphinxhyphen{}config\sphinxhyphen{}CNVseq\sphinxhyphen{}coffee.yml} and \sphinxstylestrong{example2\sphinxhyphen{}config\sphinxhyphen{}CNVseq\sphinxhyphen{}sorghum.yml}). The configuration file \sphinxstylestrong{config\sphinxhyphen{}CNVseq.yml} contains multiple parameters to be defined by the user:
\begin{itemize}
\item {} 
\sphinxAtStartPar
\sphinxtitleref{paths}:
\begin{itemize}
\item {} 
\sphinxAtStartPar
sample names and their respective paths to \sphinxstylestrong{.bam} files

\item {} 
\sphinxAtStartPar
samples can be named as desired but the sample name must be repeated after the colon and prefixed with a \sphinxtitleref{\&} sign

\item {} 
\sphinxAtStartPar
the \sphinxtitleref{\&} prefix sign is used to reference the sample’s path in different places of the same configuration file

\item {} 
\sphinxAtStartPar
example use:

\end{itemize}

\end{itemize}

\begin{sphinxVerbatim}[commandchars=\\\{\}]
\PYG{n}{paths}\PYG{p}{:}
  \PYG{n}{mysample}\PYG{p}{:} \PYG{o}{\PYGZam{}}\PYG{n}{mysample} \PYG{o}{/}\PYG{n}{home}\PYG{o}{/}\PYG{n}{john}\PYG{o}{/}\PYG{n}{bam\PYGZus{}files}\PYG{o}{/}\PYG{n}{mysample}\PYG{o}{.}\PYG{n}{bam}
  \PYG{n}{XYZ}\PYG{o}{\PYGZhy{}}\PYG{l+m+mi}{123}\PYG{p}{:} \PYG{o}{\PYGZam{}}\PYG{n}{XYZ}\PYG{o}{\PYGZhy{}}\PYG{l+m+mi}{123} \PYG{o}{/}\PYG{n}{home}\PYG{o}{/}\PYG{n}{john}\PYG{o}{/}\PYG{n}{bam\PYGZus{}files}\PYG{o}{/}\PYG{n}{XYZ}\PYG{o}{\PYGZhy{}}\PYG{l+m+mf}{123.}\PYG{n}{bam}
  \PYG{n}{potato95}\PYG{p}{:} \PYG{o}{\PYGZam{}}\PYG{n}{potato95} \PYG{o}{/}\PYG{n}{home}\PYG{o}{/}\PYG{n}{john}\PYG{o}{/}\PYG{n}{bam\PYGZus{}files}\PYG{o}{/}\PYG{n}{potato95}\PYG{o}{.}\PYG{n}{bam}
\end{sphinxVerbatim}
\begin{itemize}
\item {} 
\sphinxAtStartPar
\sphinxtitleref{bed\_path}:
\begin{itemize}
\item {} 
\sphinxAtStartPar
path to bed file if using varying window sizes

\end{itemize}

\item {} 
\sphinxAtStartPar
\sphinxtitleref{comparisons}:
\begin{itemize}
\item {} 
\sphinxAtStartPar
comparison names with respective control and mutant samples per comparison

\item {} 
\sphinxAtStartPar
each comparison can be named as desired

\item {} 
\sphinxAtStartPar
the sample names to be used as \sphinxtitleref{control} and \sphinxtitleref{mutant} need to be prefixed by a \sphinxtitleref{*} sign

\item {} 
\sphinxAtStartPar
the \sphinxtitleref{*} prefixed sign is used to extract the sample’s path defined in the \sphinxtitleref{paths} section

\item {} 
\sphinxAtStartPar
example:

\end{itemize}

\end{itemize}

\begin{sphinxVerbatim}[commandchars=\\\{\}]
\PYG{n}{comparisons}\PYG{p}{:}
  \PYG{n}{comparison}\PYG{o}{\PYGZhy{}}\PYG{l+m+mi}{1}\PYG{p}{:}
    \PYG{n}{control}\PYG{p}{:} \PYG{o}{*}\PYG{n}{mysample}
    \PYG{n}{mutant}\PYG{p}{:} \PYG{o}{*}\PYG{n}{potato95}
  \PYG{n}{a}\PYG{o}{\PYGZhy{}}\PYG{n}{different}\PYG{o}{\PYGZhy{}}\PYG{n}{comparison}\PYG{o}{\PYGZhy{}}\PYG{l+m+mi}{278}\PYG{n}{asd}\PYG{p}{:}
    \PYG{n}{control}\PYG{p}{:} \PYG{o}{*}\PYG{n}{mysample}
    \PYG{n}{mutant}\PYG{p}{:} \PYG{o}{*}\PYG{n}{XYZ}\PYG{o}{\PYGZhy{}}\PYG{l+m+mi}{123}
\end{sphinxVerbatim}
\begin{itemize}
\item {} 
\sphinxAtStartPar
\sphinxtitleref{chromosomes}:
\begin{itemize}
\item {} 
\sphinxAtStartPar
list of chromosome names to analyze separated by commas

\item {} 
\sphinxAtStartPar
chromosome names can be extracted from a bam file’s header

\end{itemize}

\item {} 
\sphinxAtStartPar
\sphinxtitleref{parameters}:
\begin{itemize}
\item {} 
\sphinxAtStartPar
parameters used to create window sizes, thresholds, and plots

\item {} 
\sphinxAtStartPar
the parameter defaults are provided

\end{itemize}

\end{itemize}


\section{Running the RCNV\_seq\sphinxhyphen{}template Jupyter Notebook}
\label{\detokenize{index:running-the-rcnv-seq-template-jupyter-notebook}}
\begin{sphinxadmonition}{note}{Note:}
\sphinxAtStartPar
It is recommended to duplicate the \sphinxstylestrong{RCNV\sphinxhyphen{}seq\sphinxhyphen{}template} notebook and then renaming the copy before doing any edits to the notebook.
\end{sphinxadmonition}

\sphinxAtStartPar
In the \sphinxstylestrong{copy\sphinxhyphen{}number\sphinxhyphen{}analysis/jupyter\sphinxhyphen{}notebooks} directory, click on \sphinxstylestrong{RCNV\sphinxhyphen{}seq\sphinxhyphen{}template} and a new tab will open the notebook.

\sphinxAtStartPar
The notebook contains cells that are populated by text or code. Instructions are provided in the notebook to guide the user.

\sphinxAtStartPar
The notebook consists of 5 sections:
\begin{enumerate}
\sphinxsetlistlabels{\arabic}{enumi}{enumii}{}{.}%
\item {} 
\sphinxAtStartPar
User Input (MANDATORY)

\item {} 
\sphinxAtStartPar
Installing Required Libraries (optional)

\item {} 
\sphinxAtStartPar
Loading Required Libraries (MANDATORY)

\item {} 
\sphinxAtStartPar
CNV Calculations and Plotting (MANDATORY)

\item {} 
\sphinxAtStartPar
Plotting Specific Window of One Chromosome (optional)

\end{enumerate}

\sphinxAtStartPar
To run a cell, click on the corresponding cell and press \sphinxstylestrong{Ctrl + Enter} or \sphinxstylestrong{Shift + Enter}.


\section{References}
\label{\detokenize{index:references}}
\sphinxAtStartPar
\sphinxstylestrong{BMC Bioinformatics Publication}:
\begin{itemize}
\item {} 
\sphinxAtStartPar
CNV\sphinxhyphen{}seq, a new method to detect copy number variation using high\sphinxhyphen{}throughput sequencing {[}\sphinxhref{https://bmcbioinformatics.biomedcentral.com/articles/10.1186/1471-2105-10-80}{LINK}{]}

\end{itemize}

\sphinxAtStartPar
\sphinxstylestrong{GitHub repositories}:
\begin{itemize}
\item {} 
\sphinxAtStartPar
\sphinxhref{https://github.com/hliang/cnv-seq}{hliang/cnv\sphinxhyphen{}seq}

\item {} 
\sphinxAtStartPar
\sphinxhref{https://github.com/Bioconductor/copy-number-analysis/wiki/CNV-seq}{Bioconductor/copy\sphinxhyphen{}number\sphinxhyphen{}analysis}

\item {} 
\sphinxAtStartPar
\sphinxhref{https://github.com/amora197/copy-number-analysis}{amora197/copy\sphinxhyphen{}number\sphinxhyphen{}analysis}

\end{itemize}



\renewcommand{\indexname}{Index}
\printindex
\end{document}